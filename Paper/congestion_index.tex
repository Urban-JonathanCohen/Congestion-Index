\documentclass[a4paper]{jpconf}
\usepackage{graphicx}
\begin{document}
\title{Towards a city congestion index: methodological explorations using Google's Distance Matrix API}

\author{Jonathan Cohen and Jorge Gil}

\address{Chalmers University of Technology, Sven Hultins gata 6, SE-412 96, Göteborg, Sweden}

\ead{jonathan.cohen@chalmers.se}




\begin{abstract}
	All articles {\it must} contain an abstract. This document describes the  preparation of a conference paper to be published in \jpcs\ using \LaTeXe\ and the \cls\ class file. The abstract text should be formatted using 10 point font and indented 25 mm from the left margin. Leave 10 mm space after the abstract before you begin the main text of your article. The text of your article should start on the same page as the abstract. The abstract follows the addresses and should give readers concise information about the content of the article and indicate the main results obtained and conclusions drawn. As the abstract is not part of the text it should be complete in itself; no table numbers, figure numbers, references or displayed mathematical expressions should be included. It should be suitable for direct inclusion in abstracting services and should not normally exceed 200 words. The abstract should generally be restricted to a single paragraph. Since contemporary information-retrieval systems rely heavily on the content of titles and abstracts to identify relevant articles in literature searches, great care should be taken in constructing both.
\end{abstract}

\

\section{Introduction}
\indent What are the problems with congestion? Is it mentioned in UN Habitat NUA, or SDGs e.g.11?\par
\indent What are the clients of this index? Who needs such an index, and to do what?\par

\section{Background}
\indent What congestion indices exist out there? \par
\indent Practice/literature\par
\indent How are others mapping congestion (or real travel time) with geodata? (edited) \par

\section{Methodology}

\section{Data}
\indent The methodolgy detailed above The information requirements for this study are relatively low. As explained above, as the method genereates the data there 


\section{Results and discusison}
\section{Conclusion}

\end{document}


