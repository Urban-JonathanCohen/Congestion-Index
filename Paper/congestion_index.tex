\documentclass[a4paper]{jpconf}
\usepackage{graphicx}
\usepackage{booktabs}
\usepackage{siunitx}
\begin{document}
\title{Towards a city congestion index: methodological explorations using Google's Distance Matrix API}

\author{Jonathan Cohen and Jorge Gil}

\address{Chalmers University of Technology, Sven Hultins gata 6, SE-412 96, Göteborg, Sweden}

\ead{jonathan.cohen@chalmers.se}




\begin{abstract}
	All articles {\it must} contain an abstract. This document describes the  preparation of a conference paper to be published in \jpcs\ using \LaTeXe\ and the \cls\ class file. The abstract text should be formatted using 10 point font and indented 25 mm from the left margin. Leave 10 mm space after the abstract before you begin the main text of your article. The text of your article should start on the same page as the abstract. The abstract follows the addresses and should give readers concise information about the content of the article and indicate the main results obtained and conclusions drawn. As the abstract is not part of the text it should be complete in itself; no table numbers, figure numbers, references or displayed mathematical expressions should be included. It should be suitable for direct inclusion in abstracting services and should not normally exceed 200 words. The abstract should generally be restricted to a single paragraph. Since contemporary information-retrieval systems rely heavily on the content of titles and abstracts to identify relevant articles in literature searches, great care should be taken in constructing both.
\end{abstract}

\

\section{Introduction}
\indent What are the problems with congestion? Is it mentioned in UN Habitat NUA, or SDGs e.g.11?\par
\indent What are the clients of this index? Who needs such an index, and to do what?\par

\section{Background}
\indent What congestion indices exist out there? \par
\indent Practice/literature\par
\indent How are others mapping congestion (or real travel time) with geodata? (edited) \par

\section{Research Aim}
\indent As mentioned above, traffic congestion is an urban problem that deteriorate the positive benefits of urban life, therefore is imperative to understand the problem in depth and look at it from different angles. Traditional congestion measurements are found to be costly and through practice or research different methods have been explored to overcome this issue. \par
\indent One avenue of research understands congestion as a time and location specific problem, which occurs in a road segment and that its consequences are suffered from those in the immediate surroundings. A second understanding of the problem deals with the individuals who are affected by traffic. \par
\indent Focusing on this second interpretation of the problem, the aim of this research is to provide a methodology to spatialize congestion and expose how traffic congestion is distributed across an urban area. The map of congestion will provide local authorities and planners with spatial information about the city and consequently provide and measure how the problem has been evolving over time and space.  \par



\section{Methodology}
At its core the methodology described in this section details a process of collecting and processing data from an internet service for traffic estimation and routing. The results presented here correspond to data extracted from the Distance Matrix API form Google Inc., but the exact same exercise can be done with other services such as HERE or TOMTOM.
In order to understand how congestion is distributed across an urban area and potentially identify who is suffering from it, this method starts by selecting the study area and identifying the boundaries of the analysis. 



As a proof of concept, to demostrate how the method could be used in practice, four European cities of different size and structure where selected as test arenas.  \par


\section{Data}
\indent The methodology detailed above The information requirements for this study are relatively low. As explained above, as the method genereates the data there 


\begin{table}[ht]	
	\caption {Descriptive Statistics}
	\label{Results}
	\centering
	\begin{tabular}{rlrrrrrr}
		\hline
		& Time Difference (mins) & Routes & N & Mean & S.D. & Min & Max \\ 
		\hline
		1 & Amsterdam	& 30,564 & 15,282 & 8.22 	& 4.54 & -1.37 & 39.62 \\ 
		2 & Glasgow 	& 33,512 & 16,756 & 10.99 	& 5.39 & -0.25 & 34.08 \\ 
		3 & Goteborg 	& 29,248 & 14,624 & 4.89 	& 2.35 & -1.25 & 14.78 \\ 
		4 & Lisbon 		& 25,870 & 12,935 & 15.21 	& 7.26 & -0.33 & 40.67 \\
		
		\hline
		& TTI & Routes &  N & Mean & S.D. & Min & Max \\ 
		\hline
		1 & Amsterdam & 30,564 & 15,282 & 1.84 & 5.08 & 0.79 & 397.17 \\ 
		2 & Glasgow   & 33,512 & 16,756 & 1.93 & 0.33 & 0.93 & 3.73 \\ 
		3 & Goteborg  & 29,248 & 14,624 & 1.38 & 0.17 & 0.86 & 2.42 \\ 
		4 & Lisbon    & 25,870 & 12,935 & 2.23 & 0.50 & 0.92 & 5.26 \\ 
		\hline
	\end{tabular}
\end{table}




\section{Results and discusison}





\begin{table}[ht]	
	\caption {Descriptive Statistics}
	\label{Results}
	\centering
	\begin{tabular}{rlrrrrrr}
		\hline
		& Time Difference (mins) & Cells & Mean & S.D. & Min & Max \\ 
		\hline
		1 & Amsterdam 	& 131 & 8.57  & 2.39 & 5.38 & 19.03 \\ 
		2 & Glasgow 	& 136 & 11.15 & 1.88 & 7.66 & 15.64 \\ 
		3 & Goteborg 	& 123 & 4.92  & 0.99 & 3.22 & 8.06  \\ 
		4 & Lisbon 		& 119 & 15.34 & 2.67 & 9.74 & 21.57 \\ 
		
		\hline
		& TTI & Cells & Mean & S.D. & Min & Max \\ 
		\hline
		1 & Amsterdam 	& 131 & 1.74 & 0.23 & 1.36 & 2.61 \\ 
		2 & Glasgow 	& 136 & 1.94 & 0.14 & 1.60 & 2.31 \\ 
		3 & Goteborg	& 123 & 1.38 & 0.09 & 1.21 & 1.69 \\ 
		4 & Lisbon 		& 119 & 2.21 & 0.16 & 1.70 & 2.57 \\ 
		\hline
	\end{tabular}
\end{table}





\section{Conclusion}

\end{document}


